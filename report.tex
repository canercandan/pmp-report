\documentclass[a4paper,oneside,12pt]{article}

\usepackage[francais]{babel}
\usepackage[utf8]{inputenc}
%\usepackage[T1]{fontenc}
\usepackage{url}
\usepackage{longtable,geometry}
\usepackage{emp}

\ifx\pdftexversion\undefined
\usepackage[dvips]{graphicx}
\else
\usepackage[pdftex]{graphicx}
\DeclareGraphicsRule{*}{mps}{*}{}
\fi

\pagestyle{headings}
\geometry{dvips,a4paper,margin=1.0in}

\title{Parallelisation et modèles de programmation\\Exercices}
\author{Caner Candan, M2 MIHP}
\date\today

\begin{document}
\begin{empfile}

\begin{empcmds}
input metauml;
\end{empcmds}

\maketitle

\tableofcontents

\newpage

\section{Produit matriciel}

\subsection{Enoncé}

\textit{Matrice: Décrire un algorithme PRAM pour le modèle EREW qui utilise $O(n^3)$ processeurs pour multiplier deux matrices de taille $n*n$.\footnote{\url{http://fr.wikipedia.org/wiki/Produit_matriciel}}}


\begin{displaymath}
c_{ij} = \sum_{k=0}^n a_{ik} b_{kj}
\end{displaymath}

% How To Embed UML in LaTeX
% http://www.cs.bgu.ac.il/~gwiener/software-engineering/how-to-embed-uml-in-latex/

\section{Design}

Tous les exercices ont été developpés en C++ en utilisant le paradigme de programmation objet et le design qui suit a été constitué pour regrouper les différents composants de calculer utilisés.

\begin{center}
\begin{emp}[classdiag](20, 20)
Class.A("Point")
       ("+x: int",
        "+y: int") ();

Class.B("Circle")
       ("radius: int")
       ("+getRadius(): int",
        "+setRadius(r: int):void");

topToBottom(45)(A, B);

drawObjects(A, B);

clink(aggregationUni)(A, B)
\end{emp}

\end{center}

\end{empfile}
\end{document}
