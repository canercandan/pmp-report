\section{Produit matriciel}

\subsection{Enoncé}
\textit{Matrice: Décrire un algorithme PRAM pour le modèle EREW qui utilise $O(n^3)$ processeurs pour multiplier deux matrices de taille $n*n$.\footnote{$c_{ij} = \sum_{k=0}^n a_{ik} b_{kj}$: \url{http://fr.wikipedia.org/wiki/Produit_matriciel}}}

\subsection{Solution}
Une solution consisterait à diffuser dans un premier temps toutes les données des matrices à multiplier à chaque processus puis de faire le produit des matrices en prenant le soin que chaque processus utilise les données qui leur est propre.

\subsubsection{La diffusion}
En respectant le modèle EREW, les matrices $A$ et $B \in K_{n*n}$ sont copiées, dans un premier temps, $n$ fois dans les cubes $A'$ et $B' \in K_{n*n*n}$. Ceci permet de reserver une dimension du cube à un processus et éviter toute lecture concurrente.\\

L'algorithme \ref{diffusion} présente le pseudo-code de la fonction de diffusion qui prend comme donnée en entrée une matrice $M \in K_{n*n}$ et retourne un cube $M' \in K_{n*n*n}$ . Elle est exprimée en 2 étapes. La première consiste à affecter, en utilisant $O(n^2)$ processeurs, chaque point de la matrice dans la première dimension du cube. La deuxième étape effectue la copie en diffusion\footnote{Hot potato routing : \url{http://en.wikipedia.org/wiki/Hot_potato_routing}} de chaque point de la matrice situé à la première dimension du cube vers les autres dimensions du cube en utilisant $O(\log{n})$ processeurs et pour une matrice entière en $O(n^2 * \log{n})$.

\incmargin{1em}
\begin{algorithm}[here]
  \dontprintsemicolon
  \label{diffusion}
  \Donnees{$M \in K_{n*n}, n \in N$}
  \Res{$M' \in K_{n*n*n}$}
  \Deb{
    \textit{parallèle}\;
    \Pour{$i\leftarrow 0$ \KwA $n$}{
      \textit{parallèle}\;
      \Pour{$j\leftarrow 0$ \KwA $n$}{
        $m'(0,i,j)\leftarrow m(i,j)$
      }
    }

    \textit{parallèle}\;
    \Pour{$s\leftarrow 0$ \KwA $\log_2{n}$}{
      \textit{parallèle}\;
      \Pour{$h\leftarrow 0$ \KwA $s^2$}{
        \textit{parallèle}\;
        \Pour{$i\leftarrow 0$ \KwA $n$}{
          \textit{parallèle}\;
          \Pour{$j\leftarrow 0$ \KwA $n$}{
            $m'(2^s + h,i,j)\leftarrow m'(h,i,j)$
          }
        }
      }
    }
  }
  \caption{Copie de matrice en diffusion}
\end{algorithm}
\decmargin{1em}

\subsubsection{Le produit}

En appliquant le modèle EREW, le produit matriciel prend en paramètre les cubes $A$ et $B \in K_{n*n*n}$ tel que chaque processus utilise sa dimension de cube respective pour lire les données de $a_{ij}$ et $b_{ij}$.\\

L'algorithme \ref{produit} présente le pseudo-code de la fonction de produit matriciel qui prend comme donnée en entrée les cubes $A$ et $B \in K_{n*n*n}$ et retourne une matrice $C \in K_{n*n}$. L'affectation des valeurs dans la matrice $C$ se fait en $O(n^2)$ et le produit de $c_{ij}$ se fait en $O(n)$.

\incmargin{1em}
\begin{algorithm}[here]
  \dontprintsemicolon
  \label{produit}
  \Donnees{$A,B \in K_{n*n*n}, n \in N$}
  \Res{$C \in K_{n*n}$}
  \Deb{
    \textit{parallèle}\;
    \Pour{$i\leftarrow 0$ \KwA $n$}{
      \textit{parallèle}\;
      \Pour{$j\leftarrow 0$ \KwA $n$}{
        $c(i,j)\leftarrow \sum_{k=0}^n a(p,i,k) * b(p,k,j)$
      }
    }
  }
  \caption{Produit matriciel}
\end{algorithm}
\decmargin{1em}

\subsubsection{L'implémentation}

Après avoir décrit la fonction de diffusion et le produit matriciel respectant tous deux le modèle EREW, nous allons voir l'implementation d'un produit de deux matrices $A$ et $B \in K_{n*n}$ produisant une troisième matrice $C \in K_{n*n}$.\\

L'algorithme \ref{implementation} décrit l'implémentation.

\incmargin{1em}
\begin{algorithm}[here]
  \dontprintsemicolon
  \label{implementation}
  \Donnees{$A,B \in K_{n*n}, n \in N$}
  \Res{$C \in K_{n*n}$}
  \Deb{
    $A'\leftarrow diffusion(A,n)$\;
    $B'\leftarrow diffusion(B,n)$\;
    $C\leftarrow produit(A',B')$\;
  }
  \caption{Implémentation du produit matriciel}
\end{algorithm}
\decmargin{1em}
