\subsubsection{Conclusion}

Pour un processeur à 2 coeurs, selon la complexité du problème utilisée, nous pouvons observer par la mesure du speedup que les ressources sont utilisées au maximum\footnote{cpu-bound} pour de grandes tailles d'échantillon. Les petites tailles d'échantillon restant dominées par une exécution séquentielle.\\

Pour un processeur à 8 coeurs, en hyperthreading, notre première observation consiste à montrer les limites de l'hyperthreading, en effet nous utilisons pas plus de 4 coeurs. Et pour rejoindre ce qui a été dit sur 2 coeurs, un plus grand nombre de petites tailles d'échantillon reste dominée par une exécution séquentielle.\\

La différence entre les deux complexités de problème s'observe dans la mesure de la dynamicité. En effet pour un problème en $O(1)$, la planification statique en OpenMP permet d'avoir une meilleur efficacité de travail tandis que en $O(n)$, la planification dynamique apporte quelque amélioration.\\

Pour résumer nous avons comparé par nos mesures les 2 types de planification de tâches en OpenMP, les 2 complexités de problèmes que nous serons amené à paralléliser et finalement sur 2 types de processeurs différents.\\

Un point que je n'ai pas eu le temps d'éclaircir que j'invite à regarder est l'utilisation de l'auto parallélisation intégré au compilateur GCC\footnote{Automatic Parallelization : \url{http://gcc.gnu.org/wiki/openmp}}. Il s'agit d'un niveau d'optimisation qui consiste à vérifier la non dépendance des données dans les boucles et les paralléliser.
